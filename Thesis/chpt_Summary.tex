This chapter provides a brief summary of the HTSLAM approach. A comparison
with other existing mapping techniques is given. 


\section{Summary of HTSLAM}

\SILENT{1 page, overview of HTSLAM (point form?)}

HTSLAM is a novel mapping technique that addresses some of the
limitations of traditional mapping approaches. HTSLAM decomposes the
environment into a number of interconnected regions. Each region is
mapped separately. There is no global reference frame in HTSLAM, only
metric relationships between neighbouring regions are stored in the
HTSLAM map. Local mapping is performed using a particle filter based
approach called FastSLAM \cite{fastslam, fastslam2}. The particle filter
allows HTSLAM to deal with mapping ambiguities in a rigorous
probabilistic framework. The advantage of a particle filter over a more
traditional EKF approach becomes especially apparent when performing
loop closing. In a general case of unknown data associations
(non-unique landmarks), loop closing gives rise to a multi-modal
probability density over robot pose, due to ambiguities arising in
data association decisions. The EKF is incapable of dealing with
multi-modal distributions directly. As a result one has to run
a separate EKF for each competing hypothesis, this however can be
computationally expensive. It is not practical to create new EKF for
every ambiguous data association decision, as the number of filters
running in parallel will increase exponentially. A pruning mechanism
is required, however comparing two EKF mapping filters is not an
obvious task. Generally one needs to define some heuristic
comparison criteria, and some thresholds to overcome this problem.

In contrast to the EKF, particle filters deal with multi-modal
distributions natively. HTSLAM exploits this ability of particle
filters, to run multiple mapping hypothesis in parallel. There is no
need to define fancy comparison metrics to prune out invalid
hypothesis. Pruning of unlikely hypothesis happens automatically as
part of a resampling step of the particle filter. 


\section{Comparison}

\subsection{Global Mapping Approaches}
%Short!

%\subsubsection{EKF SLAM}

Without doubt, the most common mapping approach to date is EKF based
SLAM. It is a well-understood technique with neat mathematical
background. The computational complexity of a pure EKF SLAM is
quadratic in the number of landmarks. However it has been shown that
the computational complexity can be reduced to almost constant time,
by exploiting locality of observations \cite{Thrun03d,guivant04}.
These techniques still require merging of the local information back
into a global map once in a while, this update step is quadratic in
the number of landmarks.

EKF makes some seriously limiting assumptions \cite{fixme}. In
particular, it is assumed that the data association is
unambiguous. This is generally false even for well behaved sensors
like laser range scanners. Batch data association techniques like
\cite{neira01:_data_assoc_stoch_mappin_using,
tardos02:_mappin_local_indoor_envir_using_sonar_data} can reduce the
ambiguity of data association, but it is impossible to eliminate this
problem completely. It is impossible to undo incorrect data
association in EKF, as a result EKF is quite sensitive to data
association errors. Loop closing in particular is a very difficult
problem for EKF SLAM, due to the inability to deal with ambiguous data
association.

EKF SLAM assumes that noise in sensor and motion models can be
approximated with a Gaussian. Such an approximation works reasonably
well for small maps. The effect of approximation becomes more apparent
in larger environments. Generally the sensor range is finite. The
robot is able to observe only a small local region around its current
pose. As the robot moves further away from the starting point, the
uncertainty of the robot pose increases, because it cannot observe
landmarks near the origin, and relocalise with respect to
them. Uncertainty in the pose propagates to the new landmarks being
added to the map. Maps produced by EKF SLAM usually have highly
certain landmarks near the origin, while landmarks far away have much
larger uncertainties. In theory the map will converge as number of
observations increases to infinity. \NOTE{this is not exactly true,
since there is no proof that EKF should converge, however it does so
in practice.} In practice increasing uncertainty and accumulation of
approximation errors is a huge problem, as it makes data association
more difficult, or even impossible, leading to inconsistent maps and
divergence.


%\subsubsection{FastSLAM and FastSLAMII}

FastSLAM is a more recent mapping technique based on a modified
Rao-Blackwellised particle filter \cite{fastslam, fastslam2}. It
avoids some of the problems of an EKF approach. The computational
complexity of FastSLAM is logarithmic in the number of landmarks $N$,
scaled by the number of particles $K$,$O(K \log N)$. Being a particle
based approach FastSLAM can model multi-modal probability distribution
functions (PDF). FastSLAM can also deal with ambiguous data
associations \cite{Montemerlo2003}, since every particle is free to
choose data association most compatible with it's state. Furthermore
new particles can be added when data association is
ambiguous. FastSLAM deals with the loop closing problem seamlessly
(given there are enough particles in the right place). Like the EKF,
FastSLAM linearises the observation model,and assumes thae errors are
Gaussian. However the motion model does not need to be linear, as a result
non-linear behaviours like wheel slippage can be modelled
probabilistically. FastSLAM was shown to work even when no odometry
information is available \cite{fastslam}.

Like the EKF approach FastSLAM has one global reference frame, as a
result it suffers from similar problems when operating in a large
environment. Increasing uncertainty as the robot moves away from the
origin, leads to the particle deprivation problem. The higher the
uncertainty the more particles are needed to model it accurately.
Approximation errors also tend to accumulate and become more obvious.
As a result the performance of the filter degrades and this in turn
can lead to inconsistent maps. FastSLAM 2.0 \cite{fastslam2} deals
with the particle deprivation problem by merging observation and
odometry information during the motion propagation stage of the filter,
effectively generating samples where they are needed most. However
this just decreases the effect of the particle deprivation problem and
does not eliminate it entirely.

HTSLAM is based on FastSLAM, so like FastSLAM it is capable of
modelling multi-modal PDFs, which in turn allows the system to capture
ambiguity arising from data association decisions. Since HTSLAM
only updates a small local region (or some times several small local
regions) at any given time, the computational complexity is
bounded. HTSLAM provides a more robust method for loop closing. In
HTSLAM loop closing can be postponed until enough information
supporting it is available. In fact it is possible to post-pone loop
detection and closing until mapping is complete and update the map at
a later time. In contrast EKF and FastSLAM have to rely on observation
to map correspondence alone when closing the loop.

The main advantage of HTSLAM is scalability. HTSLAM performance does
not degrade as the robot moves further away from the origin; the
quality of the local maps is a function of the environment and the
sensors, and is independent of the distance from the origin. Assuming
a sensible exploration strategy that avoids unnecessary long loops, or
venturing into feature-less spaces, the area that can be covered is
only limited by the robot storage capacity. A smart implementation
that swaps out unused maps from memory to disk can indeed cover large
areas with this approach.

\subsection{Atlas}

HTSLAM and Atlas share a similar hybrid map structure. Both approaches
do without a global reference frame and instead rely on local metric
information only. There are also significant differences between the
two approaches

\begin{itemize}
\item Choice of the local mapping module.
\item Map transition process.
\item Loop closing procedure.
\item Representation of uncertainty in coordinate transformations
  between map frames.
\end{itemize}

Atlas uses the EKF as its local mapping module, while HTSLAM uses
FastSLAM. FastSLAM has a number of advantages over EKF. FastSLAM is
computationally more efficient and also provides a richer
representation of uncertainty as it can model multi-modal
distributions, while the EKF is restricted to a single mode. For a local
mapping module computational complexity is not as important, as map
size is bounded. The EKFs inability to handle multi-modal distributions
make it unusable in some environments, even if the size of the map is
limited. In the case of a cluttered environment, data association can be
highly ambiguous leading to data association errors. Incorrect data
association is known to lead to inconsistent maps when using EKF.
FastSLAM can deal with data association ambiguity by sampling over all
plausible data association decisions
\cite{fastslam,Montemerlo2003,nieto2003}. The authors of Atlas do mention
the possibility of using mapping modules other than EKF, however as of
time of the writing I am not aware of any such implementations.

%Transitions and Local Map region boundary.

HTSLAM maintains map extent information for every local map. This
extra data allows HTSLAM to perform map transitions in a simple
way. In HTSLAM the estimate of the robot pose in a current reference
frame is all that is needed to make a decision on whether to stay in
the current frame, perform a transition into a neighbouring region or
to start a new map. In contrast, Atlas runs multiple localisation
hypothesis in all neighbouring regions in order to determine which
region provides a better explanation to current observations. Atlas
uses some quality metric to judge the fitness of each
alternative. Generally when traversing a region that has been mapped
previously, the robot is well localised, and so there is no ambiguity with
regards to which local map the robot should be updating. It is therefore
unnecessary to run multiple hypotheses in such a situation. HTSLAM only
generates multiple hypothesis if there is significant ambiguity in the
robot pose.

%write a paragraph or so on how HTSLAM is more elegant
%Multiple Hypotheses

Both Atlas and HTSLAM use multiple hypothesis to prove or disprove
the validity of loop closing. Atlas uses some performance metrics to
compare different hypothesis, and to prune out the weak ones. In
HTSLAM mapping hypothesis are compared directly with each other within
a common particle filter framework. Importance
sampling is used to prune out the unlikely hypotheses. There is no
need to invent performance metrics. It is therefore the author's belief
that HTSLAM provides a more formal approach to management of
uncertainty arising due to loop closing than Atlas.


%\subsubsection{Rigorous/Proper/Formal Approach}
%Cross-correlations!

Atlas does not maintain correlations between local maps, nor does it
keep correlations between maps and transitions. By construction the
transition is a dependent variable of a local map. In HTSLAM this
dependence is captured in a sample.


\SILENT{
A. Transition from current map to new map is by construction a
dependent variable of current map.
B. They update transitions based on localisation within the two maps,
localisation depends on the maps, that makes transition dependent on
both maps.
C. The process of ``seeding'' pose clearly introduces the dependence,
and no matter how long they run localisation afterwards, that
dependence is not going to go away.
}



% LocalWords:  HTSLAM FastSLAM EKF resampling relocalise FastSLAMII Rao
% LocalWords:  Blackwellised odometry
