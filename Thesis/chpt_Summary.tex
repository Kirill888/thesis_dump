This chapter provides a brief summary of the HTSLAM approach. A comparison
with other existing mapping techniques is given. 


\section{Summary of HTSLAM}

\SILENT{1 page, overview of HTSLAM (point form?)}

HTSLAM is a novel mapping technique that addresses some of the
limitations of traditional mapping approaches. HTSLAM decomposes the
environment into a number of interconnected regions. Each region is
mapped separately. There is no global reference frame, only
metric relationships between neighbouring regions are stored in the
HTSLAM map. Local mapping is performed using a particle filter based
approach called FastSLAM \cite{fastslam, fastslam2}. The particle filter
allows HTSLAM to deal with mapping ambiguities in a rigorous
probabilistic framework.


In a general case of unknown data associations (non-unique landmarks),
probability density over robot pose is often multi-modal, due to
ambiguities arising in the data association decisions. Particle filters
deal with multi-modal distributions natively. HTSLAM benefits from this
ability of particle filters. This advantage is especially apparent when
closing large cycles. When closing the loop HTSLAM runs multiple mapping
hypothesis in parallel. Pruning of unlikely hypothesis happens
automatically as part of a resampling step of the particle filter.



\section{Comparison}

\subsection{Global Mapping Approaches}

Global mapping approaches include EKF SLAM and FastSLAM. These methods
and their limitations were described in detail in
Chapter~\ref{chpt:LiteratureReview}.

The main advantage of HTSLAM over global methods is scalability. HTSLAM
performance does not degrade as the robot moves further away from the
origin; the quality of the local maps is a function of the environment
and the sensors, and is independent of the distance from the
origin. Assuming a sensible exploration strategy that avoids unnecessary
long loops, or venturing into feature-less spaces, the area that can be
covered is only limited by the robot storage capacity. A smart
implementation that swaps out unused maps from memory to disk can indeed
cover large areas with this approach.

Since HTSLAM only updates a small local region (or some times
several small local regions) at any given time, the computational
complexity is bounded.

HTSLAM provides a robust method for loop closing. In HTSLAM loop
closing can be postponed until enough information supporting it is
available. In fact it is possible to post-pone loop detection and
closing until mapping is complete and update the map at a later time. In
contrast, EKF and FastSLAM have to rely on observation to map
correspondence alone when closing the loop.


\subsection{Atlas}

HTSLAM and \Atlas\ share a similar hybrid map structure. Both approaches
do without a global reference frame and instead rely on local metric
information only. There are also significant differences between the
two approaches

\begin{itemize}
\item Choice of the local mapping module.
\item Map transition process.
\item Loop closing procedure.
\item Representation of uncertainty in coordinate transformations
  between map frames.
\end{itemize}

\Atlas\ uses the EKF as its local mapping module, while HTSLAM uses
FastSLAM. FastSLAM has a number of advantages over EKF. FastSLAM is
computationally more efficient and also provides a richer representation
of uncertainty as it can model multi-modal distributions, while the EKF
is restricted to a single mode. For a local mapping module computational
complexity is not as important, as map size is bounded. The EKFs
inability to handle multi-modal distributions makes it unusable in some
environments, even if the size of the map is limited.  In high clutter
environments data association errors are more likely. Incorrect data
association is known to lead to inconsistent maps when using
EKF. FastSLAM can deal with data association ambiguity by sampling over
all plausible data association decisions
\cite{fastslam,Montemerlo2003,nieto2003}. The authors of \Atlas\ do
mention the possibility of using mapping modules other than EKF, however
as of time of the writing no such implementations have been reported.

%Transitions and Local Map region boundary.

HTSLAM maintains map extent information for every local map. This extra
data allows HTSLAM to perform map transitions in a simple way. In HTSLAM
the estimate of the robot pose in a current reference frame is all that
is needed to make a decision on whether to stay in the current frame,
perform a transition into a neighbouring region or to start a new
map. In contrast, \Atlas\ runs multiple localisation hypothesis in all
neighbouring regions in order to determine which region provides a
better explanation to current observations. \Atlas\ uses some quality
metric to judge the fitness of each alternative. Generally, when
traversing a region that has been mapped previously, the robot is well
localised, there is no ambiguity with regards to which local map
the robot should be updating. It is therefore unnecessary to run
multiple hypotheses in such a situation. HTSLAM only generates multiple
hypothesis if there is significant ambiguity in the robot pose.

%write a paragraph or so on how HTSLAM is more elegant
%Multiple Hypotheses

Both \Atlas\ and HTSLAM use multiple hypothesis to prove or disprove the
validity of loop closing. \Atlas\ uses some performance metrics to
compare different hypothesis, and to prune out the weak ones. In HTSLAM
mapping hypothesis are compared directly with each other within a common
particle filter framework. Importance sampling is used to prune out the
unlikely hypotheses. There is no need to invent performance
metrics. HTSLAM provides a more formal approach to management of
uncertainty arising due to loop closing.


%\subsubsection{Rigorous/Proper/Formal Approach}
%Cross-correlations!

\Atlas\ does not maintain correlations between local maps, nor does it
keep correlations between maps and transitions. By construction the
transition is a dependent variable of a local map. In HTSLAM this
dependence is captured in a sample.


%\SILENT{
%A. Transition from current map to new map is by construction a
%dependent variable of current map.
%B. They update transitions based on localisation within the two maps,
%localisation depends on the maps, that makes transition dependent on
%both maps.
%C. The process of ``seeding'' pose clearly introduces the dependence,
%and no matter how long they run localisation afterwards, that
%dependence is not going to go away.
%}



% LocalWords:  HTSLAM FastSLAM EKF resampling relocalise FastSLAMII Rao
% LocalWords:  Blackwellised odometry
